\documentclass[12pt]{article}
\usepackage[twocolumn,landscape,a4paper,margin=15mm,columnsep=30mm]{geometry}
\usepackage[dutch]{babel}

\usepackage{graphicx}
\usepackage{parskip}
\usepackage{color}
\usepackage{PTSansNarrow} 
\usepackage[T1]{fontenc}
\usepackage{array}
\renewcommand\familydefault\sfdefault

\pagestyle{empty}

\renewcommand\emph[1]{{\slshape#1}}

\definecolor{DolRed}{rgb}{0.92,0.11,0.15} % #eb1c26
\definecolor{DolGray}{rgb}{0.82,0.82,0.82}

\newcommand\DolHeader[1]{
\includegraphics[width=.45\columnwidth]{dol_logo_dark.png}\hfill\parbox[b]{.5\columnwidth}{#1}
\DolHline \\
{\bf Gaffelstraat 61B \DolSep Toegang gratis \DolSep Inloop: 19:30 uur \DolSep Start: 20:00 uur \DolSep Eind: 22:00 uur} \\
\DolHline
\vfill
}

\newcommand\DolHeaderLarge[1]{
\includegraphics[width=.45\columnwidth]{dol_logo_dark.png}\hfill\parbox[b]{.54\columnwidth}{#1}
\DolHline \\
{\bf Gaffelstraat 61B \DolSep Toegang gratis \DolSep Inloop 19:30 \DolSep Start 20:00 \DolSep Eind 22:00 } \\
\DolHline
}

\newcommand\DolHeaderLargeEn[1]{
\includegraphics[width=.45\columnwidth]{dol_logo_dark.png}\hfill\parbox[b]{.54\columnwidth}{#1}
\DolHline \\
{\bf Gaffelstraat 61B \DolSep Free entrance \DolSep Open 19:30 \DolSep Start 20:00 \DolSep End 22:00 } \\
\DolHline
}

\newcommand\DolHeaderEn[1]{
\includegraphics[width=.45\columnwidth]{dol_logo_dark.png}\hfill\parbox[b]{.5\columnwidth}{#1}
\DolHline \\
{\bf Gaffelstraat 61B \DolSep Free entrance \DolSep Doors open: \ 19:30 \DolSep Start: \ 20:00 \DolSep End: \ 22:00} \\
\DolHline
\vfill
}

\newcommand\DolFooter{
\vfill
\colorbox{DolRed}{\parbox{\columnwidth}{\ \parbox[b]{.88\columnwidth}{\color{white}\raggedright
{\bf `Denken over links', een maandelijkse lezingenserie in het SP-pand aan
de Gaffelstraat. Doe mee, denk mee! Iedereen is welkom.} \\[-7pt]
\hfill denkenoverlinks.nl\\[.2em]}
\hfill \includegraphics[scale=.4]{qr.png}}}
}

\newcommand\DolFooterLarge{
\vfill
\colorbox{DolRed}{\parbox{\columnwidth}{\ \ \ \parbox[b]{.84\columnwidth}{\color{white}\raggedright
{\bf `Denken over links', een maandelijkse lezingenserie in het SP-pand aan
de Gaffelstraat. Kom langs en denk mee!} \hfill denkenoverlinks.nl \\[4pt]}
\hfill \includegraphics[scale=.45]{qr.png}}}
}

\newcommand\DolFooterLargeEn{
\vfill
\colorbox{DolRed}{\parbox{\columnwidth}{\ \ \ \parbox[b]{.84\columnwidth}{\color{white}\raggedright
{\bf `Denken over links' is a monthly lecture series in the SP rooms in Gaffelstraat.
Come along and be left thinking!} \hfill denkenoverlinks.nl \\[4pt]}
\hfill \includegraphics[scale=.45]{qr.png}}}
}

\newenvironment{DolQuote}
{%
  \begin{quote}
  \begin{picture}(0,0)\put(-20,-34){\fontsize{2cm}{1em}\selectfont\bf\color{DolGray}``}\end{picture}
  \raggedright
  \setlength\parindent{1.5ex}
}{%
  \end{quote}
}

\newcommand\largeblack[2]{{\fontsize{#1}{1em}\selectfont#2}\par}

\newcommand\DolHline{\textcolor{DolRed}{\rule{\columnwidth}{6pt}}}
\newcommand\DolBullet{\textcolor{DolGray}{$\bullet$}}
\newcommand\DolSep{\hfill\DolBullet\hfill}
\newcommand\DolMega[1]{{\fontsize{1.15cm}{1em}\selectfont\bf#1}\par}
\newcommand\DolHuge[1]{\textcolor{DolRed}{\Huge\bf#1}\par}
\newcommand\DolLarge[1]{\textcolor{DolRed}{\Large\bf#1}\par}
\newcommand\DolBox[1]{\colorbox{DolRed}{\parbox{\columnwidth}{\centering\parbox{.95\columnwidth}{\rule{0pt}{13pt}\color{white}\large\bf#1\rule[-7pt]{0pt}{0pt}}}}}


\begin{document}

\DolHeaderLarge{ \raggedleft
\largeblack{6.5mm}{
{\bf Donderdag 25 september} \\[7pt]
Rotterdam \\[7pt]}
}

\emph{Onder het motto `denken over links' biedt SP Rotterdam een podium aan
inspirerende sprekers over diverse thema's. Een lezingenserie waarin het denken
over hedendaagse sociale vraagstukken wordt geprikkeld en gescherpt. \ Deze
maand:}

\vfill
\vfill

\largeblack{13.5mm}{\bf NIEUW LINKS IN EUROPA}

\vfill

Een Engelstalige editie, met Spaanse en Griekse sprekers over:

\textbf{\large Podemos}
\begin{quote}
  Spaanse politieke partij ``we kunnen'', opgericht door progressieve
  activisten uit de 15-M beweging. Sinds de Europese verkiezingen van 25 mei
  goed voor 5 zetels in het Europarlement.
\end{quote}

\textbf{\large Syriza}
\begin{quote}
  De Griekse ``coalitie van radicaal links'', ontstaan uit 13 oorspronkelijk
  onafhankelijke groeperingen. Bezet momenteel 71 zetels in het nationale en 6
  zetels in het Europarlement.
\end{quote}

\vfill

Wat zijn de perspectieven van deze (relatief) nieuwe politieke bewegingen? Hoe verklaren 
ze hun succes? Hoe verhoudt het activisme op straat zich tot de
parlementaire politiek? Deze en andere vragen in een speciale internationale avond!

\DolHline

\newpage

\DolMega{\textcolor{DolRed}{Nieuw Links in Europa}}

\vfill

\textbf{Donderdag 25 september is de aftrap van een nieuwe serie lezingen, waarin het 
denken over hedendaagse sociale vraagstukken wordt geprikkeld en gescherpt. Onder 
het motto Denken Over Links biedt de SP Rotterdam een podium voor inspirerende 
sprekers over diverse thema's als samenleven in de stad, racisme en de toekomst van 
de verzorgingsstaat. We trappen af met lezingen en discussie over de perspectieven van 
Europees links.}

\vfill

Bij de Europese verkiezingen in mei bleef de doorbraak van een Europees `links
van links' uit. Ondanks de crisis en de desastreuze bezuinigingspolitiek bleek
links nog onvoldoende in staat een geloofwaardig sociaal alternatief neer te
zetten. Maar er waren ook lichtpunten. Het Griekse Syriza, het
boegbeeld van Europees radicaal links, werd de grootste partij in Griekenland.
En het nog maar vier maanden oude Podemos, de stem van de Spaanse
\emph{Indignados}, scoorde in een klap acht procent van de stemmen in Spanje. 

Deze partijen zijn actief in landen die hard getroffen zijn door de crisis en die met grote 
werkloosheid, armoede en harde bezuinigingen te maken hebben. Zij weten met een 
radicaal programma grote groepen mensen aan zich te binden. Dat maakt nieuwsgierig 
naar het verhaal achter deze bewegingen.

Wat zijn de perspectieven van deze (relatief) nieuwe politieke bewegingen? Hoe verklaren 
ze hun succes? Hoe zit het met de verhouding tussen het activisme op straat en de 
parlementaire politiek? Deze en andere vragen komen aan de orde in twee lezingen van 
sprekers uit Spanje en Griekenland en de discussie die daarop volgt.

\vfill

\textbf{Let op: het programma is -- bij wijze van uitzondering -- Engelstalig!}

\emph{De eerstvolgende edities van Denken over Links staan in het teken van hedendaags 
racisme en discriminatie (30 oktober) en gentrificatie (27 november).}

\DolFooterLarge

\end{document}
