\documentclass[12pt]{article}
\usepackage[english]{babel}

\usepackage{graphicx}
\usepackage{parskip}
\usepackage{color}
\usepackage{PTSansNarrow} 
\usepackage[T1]{fontenc}
\usepackage{array}
\renewcommand\familydefault\sfdefault

\pagestyle{empty}

\renewcommand\emph[1]{{\slshape#1}}

\definecolor{DolRed}{rgb}{0.92,0.11,0.15} % #eb1c26
\definecolor{DolGray}{rgb}{0.82,0.82,0.82}

\newcommand\DolHeader[1]{
\includegraphics[width=.45\columnwidth]{dol_logo_dark.png}\hfill\parbox[b]{.5\columnwidth}{#1}
\DolHline \\
{\bf Gaffelstraat 61B \DolSep Toegang gratis \DolSep Inloop: 19:30 uur \DolSep Start: 20:00 uur \DolSep Eind: 22:00 uur} \\
\DolHline
\vfill
}

\newcommand\DolHeaderLarge[1]{
\includegraphics[width=.45\columnwidth]{dol_logo_dark.png}\hfill\parbox[b]{.54\columnwidth}{#1}
\DolHline \\
{\bf Gaffelstraat 61B \DolSep Toegang gratis \DolSep Inloop 19:30 \DolSep Start 20:00 \DolSep Eind 22:00 } \\
\DolHline
}

\newcommand\DolHeaderLargeEn[1]{
\includegraphics[width=.45\columnwidth]{dol_logo_dark.png}\hfill\parbox[b]{.54\columnwidth}{#1}
\DolHline \\
{\bf Gaffelstraat 61B \DolSep Free entrance \DolSep Open 19:30 \DolSep Start 20:00 \DolSep End 22:00 } \\
\DolHline
}

\newcommand\DolHeaderEn[1]{
\includegraphics[width=.45\columnwidth]{dol_logo_dark.png}\hfill\parbox[b]{.5\columnwidth}{#1}
\DolHline \\
{\bf Gaffelstraat 61B \DolSep Free entrance \DolSep Doors open: \ 19:30 \DolSep Start: \ 20:00 \DolSep End: \ 22:00} \\
\DolHline
\vfill
}

\newcommand\DolFooter{
\vfill
\colorbox{DolRed}{\parbox{\columnwidth}{\ \parbox[b]{.88\columnwidth}{\color{white}\raggedright
{\bf `Denken over links', een maandelijkse lezingenserie in het SP-pand aan
de Gaffelstraat. Doe mee, denk mee! Iedereen is welkom.} \\[-7pt]
\hfill denkenoverlinks.nl\\[.2em]}
\hfill \includegraphics[scale=.4]{qr.png}}}
}

\newcommand\DolFooterLarge{
\vfill
\colorbox{DolRed}{\parbox{\columnwidth}{\ \ \ \parbox[b]{.84\columnwidth}{\color{white}\raggedright
{\bf `Denken over links', een maandelijkse lezingenserie in het SP-pand aan
de Gaffelstraat. Kom langs en denk mee!} \hfill denkenoverlinks.nl \\[4pt]}
\hfill \includegraphics[scale=.45]{qr.png}}}
}

\newcommand\DolFooterLargeEn{
\vfill
\colorbox{DolRed}{\parbox{\columnwidth}{\ \ \ \parbox[b]{.84\columnwidth}{\color{white}\raggedright
{\bf `Denken over links' is a monthly lecture series in the SP rooms in Gaffelstraat.
Come along and be left thinking!} \hfill denkenoverlinks.nl \\[4pt]}
\hfill \includegraphics[scale=.45]{qr.png}}}
}

\newenvironment{DolQuote}
{%
  \begin{quote}
  \begin{picture}(0,0)\put(-20,-34){\fontsize{2cm}{1em}\selectfont\bf\color{DolGray}``}\end{picture}
  \raggedright
  \setlength\parindent{1.5ex}
}{%
  \end{quote}
}

\newcommand\largeblack[2]{{\fontsize{#1}{1em}\selectfont#2}\par}

\newcommand\DolHline{\textcolor{DolRed}{\rule{\columnwidth}{6pt}}}
\newcommand\DolBullet{\textcolor{DolGray}{$\bullet$}}
\newcommand\DolSep{\hfill\DolBullet\hfill}
\newcommand\DolMega[1]{{\fontsize{1.15cm}{1em}\selectfont\bf#1}\par}
\newcommand\DolHuge[1]{\textcolor{DolRed}{\Huge\bf#1}\par}
\newcommand\DolLarge[1]{\textcolor{DolRed}{\Large\bf#1}\par}
\newcommand\DolBox[1]{\colorbox{DolRed}{\parbox{\columnwidth}{\centering\parbox{.95\columnwidth}{\rule{0pt}{13pt}\color{white}\large\bf#1\rule[-7pt]{0pt}{0pt}}}}}


\begin{document}

\DolHeaderLargeEn{ \raggedleft
\largeblack{6.5mm}{
{\bf Thursday September 25} \\[7pt]
Rotterdam \\[7pt]}
}

\emph{Under the title `Left Thinking,' the Rotterdam branch of the Socialist
Party offers a podium for inspiring speakers on diverse issues. A lecture
series designed to provoke thought and sharpen opinion on present-day social
affairs. This month:}

\vfill
\vfill

\largeblack{13.5mm}{\bf EUROPE'S NEW LEFT}

\vfill

In this English language-edition, speakers from Spain and Greece will discuss:

\textbf{\large Podemos}
\begin{quote}
  A Spanish political party -- ``\emph{podemos}'' means ``we can'' -- founded by
  progressive activists from the 15-M protest movement. Since the May 25 elections
  Podemos holds 5 seats in the European Parliament.
\end{quote}

\textbf{\large Syriza}
\begin{quote}
  The Greek ``coalition of the radical left'', formed from 13 originally
  independent groups. It presently holds 71 seats in the national and 6 seats
  in the European Parliament.
\end{quote}

\vfill

What are the perspectives for these relatively new political movements? How can 
we explain their success? How does on-the-streets 
activism interact with parlia\-mentary politics? These and other questions in a
very special international evening!

\DolHline

\newpage

\DolMega{\textcolor{DolRed}{Europe's New Left}}

\vfill

\textbf{This year's edition of `Left Thinking' -- a series designed to
stimulate and focus progressive discussion on current social affairs -- kicks
off on Thursday, September 25. Presented by the Rotterdam branch of the
Socialist Party, `Denken over Links' offers a podium for inspiring and
thought-provoking speakers on such diverse issues as living together in the
city, racism and discrimination, and the future of the welfare state, amongst
many others. The 2014/15 season opens with an evening of lectures
concerning the perspectives for the Left in Europe.}

\vfill

The European elections in May didn't produce a genuine, Europe-wide
breakthrough for the authentic Left. Despite the economic crisis, and
disastrous austerity measures, the Left, broadly speaking, appears as yet
unable to present a credible social alternative. But it's not all disappointing news:
Greece's Syriza, the figurehead for Europe's radical left, became the
leading electoral force in that country; and Podemos, the voice of
Spain's \emph{indignados}, immediately won 8\% of the vote, despite having been formed
only four months beforehand.

These parties -- active in countries hit hard by the crisis, with soaring unemployment, 
deep poverty and far-reaching austerity -- have managed to attract wide support 
with their radical programmes. And there are some intriguing stories behind these 
results\dots

What are the perspectives for these relatively new political movements? How can 
we explain their success? How do they see the relationship between on-the-streets 
activism and parliamentary politics? These and other questions form the basis for 
lectures from Spanish and Greek speakers, to be followed by an 
open discussion.

\textbf{Due to the nature of the evening, the lectures and discussion will be in English.}

\emph{Coming up in the series of Left Thinking: Modern-day Racism and Discrimination
(October 30), and Urban Gentrification (November 27).}

\DolFooterLargeEn

\end{document}
