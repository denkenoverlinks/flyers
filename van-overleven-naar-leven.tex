\documentclass{article}

\usepackage[twocolumn,landscape,a4paper,margin=15mm,columnsep=30mm]{geometry}
\usepackage{graphicx}
\usepackage{parskip}
\usepackage{color}
\usepackage{PTSansNarrow} 
\usepackage[T1]{fontenc}
\usepackage[dutch]{babel}
\usepackage{array}
\renewcommand\familydefault\sfdefault

\pagestyle{empty}

\definecolor{myred}{rgb}{0.92,0.11,0.15} % #eb1c26
\definecolor{mygray}{rgb}{0.82,0.82,0.82}

\newenvironment{myquote}
{%
  \begin{quote}
  \begin{picture}(0,0)\put(-20,-34){\fontsize{2cm}{1em}\selectfont\bf\color{mygray}``}\end{picture}
  \raggedright
  \setlength\parindent{1.5ex}
}{%
  \end{quote}
}

\newcommand\largeblack[2]{{\fontsize{#1}{1em}\selectfont\bf#2}\par}

\newcommand\myhline{\textcolor{myred}{\rule{\columnwidth}{6pt}}}
\newcommand\mysep{\textcolor{mygray}{\hfill$\bullet$\hfill}}
\newcommand\mymega[1]{{\fontsize{1.15cm}{1em}\selectfont\bf#1}\par}
\newcommand\myhuge[1]{\textcolor{red}{\Huge\bf#1}\par}
\newcommand\mylarge[1]{\textcolor{red}{\Large\bf#1}\par}
\newcommand\mybox[1]{\colorbox{myred}{\parbox{\columnwidth}{\centering\parbox{.95\columnwidth}{\rule{0pt}{13pt}\color{white}\large\bf#1\rule[-7pt]{0pt}{0pt}}}}}

\newcommand\mytitle[1]{%
  \renewcommand\arraystretch{2}
  \includegraphics[width=.6\columnwidth]{dol_logo.png}\hfill\begin{tabular}[b]{@{}>{\color{red}\Huge\bf}r@{}}#1\end{tabular}\par
  \myhline \\
  {\bf Gaffelstraat 61B \mysep Toegang gratis \mysep Inloop: 19:30 uur \mysep Start: 20:00 uur \mysep Eind: 22:00 uur} \\
  \myhline
  \par\vspace{7mm}\par
}
 % packages, settings, functions

\begin{document}

\mytitle{donderdag \\ 12 dec}

\largeblack{11mm}{VAN OVERLEVEN NAAR LEVEN!}

\vfill

\textbf{Onder het motto `denken over links' biedt de SP Rotterdam in haar
nieuwe onderkomen aan de Gaffelstraat een podium voor inspirerende sprekers
over diverse thema's als samenleven in de stad, actief burgerschap, economie en
Europa. Een maandelijkse lezingenserie waarin het denken over hedendaagse
sociale vraagstukken wordt geprikkeld en gescherpt.}

\vfill

\mylarge{Sprekers}

\vfill

\textbf{\large Laura van Duin}
\begin{quote}
  Socioloog en promovendus, verbonden aan de
  Academische Werkplaats bij de Nieuwe Kans van het VU
  medisch centrum.
\end{quote}

\vfill

\textbf{\large Art-Jan van Capellen}
\begin{quote}
  Teamleider bij Stichting De Nieuwe Kans in Rotterdam.
\end{quote}

\vfill

\mybox{`Denken over links', een maandelijkse lezingenserie in het SP-pand aan
de Gaffelstraat. Doe mee, denk mee! Iedereen is welkom.}

\newpage

\myhuge{Van overleven naar leven!}

\vfill

\textbf{Kwetsbare jongvolwassenen. Jongvolwassenen met een veelheid aan problemen zoals
gebrek aan inkomen, schulden, dak- of thuisloosheid, ontbreken van
dagbesteding, geen startkwalificatie, verslavingsproblematiek, psychiatrische
problematiek en delinquent gedrag. In 2013 heeft Rotterdam met zo'n 6000
kwetsbare jongeren te maken. Dat is een substantieel deel (6\%) van de
jongvolwassenen tussen de 18 en 27 jaar.}

\vfill

Ze hebben nog een heel leven voor zich, maar als ze niet worden geholpen gaat
het gigantisch mis op allerlei fronten voor hen zelf en de samenleving. Een
deel heeft al veel hulp gehad, maar hoe help je ze nou echt vooruit met behulp
van onderzochte interventies? En wat levert effectieve hulpverlening de
samenleving op? Hoe is het eigenlijk om vanuit de praktijk met deze
jongvolwassenen te werken?

\vfill

\mylarge{Stichting De Nieuwe Kans}

Sinds 2007 heeft team De Nieuwe Kans ervaring opgedaan met het begeleiden van
``harde kern'' jongeren in Rotterdam. De Nieuwe Kans biedt een traject om
jongeren tussen de 18 en 27 jaar maatschappijvaardig te maken met behulp van
een slagvaardige aanpak. Het gaat hierbij om jongeren die met hun opleiding
zijn gestopt en (nog) niet participeren in de maatschappij middels werk of het
volgen van een opleiding. Ook een zorgtraject kan het gevolg zijn van deelname
aan De nieuwe Kans.

Een aanzienlijk deel van het programma van De Nieuwe Kans is ervaringsgericht:
wat je zelf eerder ervaren hebt, kun je met meer overtuiging en
inlevingsvermogen overbrengen op elkaar. In het programma wordt relatief veel
sport en beweging aangeboden. Dit is ontstaan vanuit de visie die De Nieuwe
Kans heeft op de mens in het algemeen en op jongeren in het bijzonder. Gezond
en verantwoord bewegen stoelt op een meerdimensionale visie. Fysieke, sociale,
emotionele en mentale aspecten spelen een bindende rol.

\vfill

\mylarge{Te gast zijn:}

\textbf{Laura van Duin},
  socioloog en promovendus, verbonden aan de Academische Werkplaats bij de
  Nieuwe Kans van het VU medisch centrum. Laura gaat in op het belang van
  onderzochte interventies en het onderzoek naar kwetsbare jongvolwassenen in
  Rotterdam.

\textbf{Art-Jan van Capellen},
  teamleider bij Stichting De Nieuwe Kans in Rotterdam. Art-Jan zal vanuit zijn
  ervaring vertellen over het werken met deze doelgroep.

\vfill

Nieuwsgierig? Kom langs en debateer met ons mee op 13 juni op de Gaffelstraat.
Aanvang 20:00, deur open 19:30. De toegang is gratis.

\myhline

\end{document}
