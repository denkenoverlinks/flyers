\documentclass[12pt]{article}
\usepackage[dutch]{babel}

\usepackage{graphicx}
\usepackage{parskip}
\usepackage{color}
\usepackage{PTSansNarrow} 
\usepackage[T1]{fontenc}
\usepackage{array}
\renewcommand\familydefault\sfdefault

\pagestyle{empty}

\renewcommand\emph[1]{{\slshape#1}}

\definecolor{DolRed}{rgb}{0.92,0.11,0.15} % #eb1c26
\definecolor{DolGray}{rgb}{0.82,0.82,0.82}

\newcommand\DolHeader[1]{
\includegraphics[width=.45\columnwidth]{dol_logo_dark.png}\hfill\parbox[b]{.5\columnwidth}{#1}
\DolHline \\
{\bf Gaffelstraat 61B \DolSep Toegang gratis \DolSep Inloop: 19:30 uur \DolSep Start: 20:00 uur \DolSep Eind: 22:00 uur} \\
\DolHline
\vfill
}

\newcommand\DolHeaderLarge[1]{
\includegraphics[width=.45\columnwidth]{dol_logo_dark.png}\hfill\parbox[b]{.54\columnwidth}{#1}
\DolHline \\
{\bf Gaffelstraat 61B \DolSep Toegang gratis \DolSep Inloop 19:30 \DolSep Start 20:00 \DolSep Eind 22:00 } \\
\DolHline
}

\newcommand\DolHeaderLargeEn[1]{
\includegraphics[width=.45\columnwidth]{dol_logo_dark.png}\hfill\parbox[b]{.54\columnwidth}{#1}
\DolHline \\
{\bf Gaffelstraat 61B \DolSep Free entrance \DolSep Open 19:30 \DolSep Start 20:00 \DolSep End 22:00 } \\
\DolHline
}

\newcommand\DolHeaderEn[1]{
\includegraphics[width=.45\columnwidth]{dol_logo_dark.png}\hfill\parbox[b]{.5\columnwidth}{#1}
\DolHline \\
{\bf Gaffelstraat 61B \DolSep Free entrance \DolSep Doors open: \ 19:30 \DolSep Start: \ 20:00 \DolSep End: \ 22:00} \\
\DolHline
\vfill
}

\newcommand\DolFooter{
\vfill
\colorbox{DolRed}{\parbox{\columnwidth}{\ \parbox[b]{.88\columnwidth}{\color{white}\raggedright
{\bf `Denken over links', een maandelijkse lezingenserie in het SP-pand aan
de Gaffelstraat. Doe mee, denk mee! Iedereen is welkom.} \\[-7pt]
\hfill denkenoverlinks.nl\\[.2em]}
\hfill \includegraphics[scale=.4]{qr.png}}}
}

\newcommand\DolFooterLarge{
\vfill
\colorbox{DolRed}{\parbox{\columnwidth}{\ \ \ \parbox[b]{.84\columnwidth}{\color{white}\raggedright
{\bf `Denken over links', een maandelijkse lezingenserie in het SP-pand aan
de Gaffelstraat. Kom langs en denk mee!} \hfill denkenoverlinks.nl \\[4pt]}
\hfill \includegraphics[scale=.45]{qr.png}}}
}

\newcommand\DolFooterLargeEn{
\vfill
\colorbox{DolRed}{\parbox{\columnwidth}{\ \ \ \parbox[b]{.84\columnwidth}{\color{white}\raggedright
{\bf `Denken over links' is a monthly lecture series in the SP rooms in Gaffelstraat.
Come along and be left thinking!} \hfill denkenoverlinks.nl \\[4pt]}
\hfill \includegraphics[scale=.45]{qr.png}}}
}

\newenvironment{DolQuote}
{%
  \begin{quote}
  \begin{picture}(0,0)\put(-20,-34){\fontsize{2cm}{1em}\selectfont\bf\color{DolGray}``}\end{picture}
  \raggedright
  \setlength\parindent{1.5ex}
}{%
  \end{quote}
}

\newcommand\largeblack[2]{{\fontsize{#1}{1em}\selectfont#2}\par}

\newcommand\DolHline{\textcolor{DolRed}{\rule{\columnwidth}{6pt}}}
\newcommand\DolBullet{\textcolor{DolGray}{$\bullet$}}
\newcommand\DolSep{\hfill\DolBullet\hfill}
\newcommand\DolMega[1]{{\fontsize{1.15cm}{1em}\selectfont\bf#1}\par}
\newcommand\DolHuge[1]{\textcolor{DolRed}{\Huge\bf#1}\par}
\newcommand\DolLarge[1]{\textcolor{DolRed}{\Large\bf#1}\par}
\newcommand\DolBox[1]{\colorbox{DolRed}{\parbox{\columnwidth}{\centering\parbox{.95\columnwidth}{\rule{0pt}{13pt}\color{white}\large\bf#1\rule[-7pt]{0pt}{0pt}}}}}


\begin{document}

\DolHeaderLarge{ \raggedleft
\largeblack{6.5mm}{
{\bf Donderdag 27 november} \\[7pt]
Rotterdam \\[7pt]}
}

\vfill
\largeblack{11.4mm}{\bf STAD ZOEKT MIDDENKLASSE}
\vfill

\begin{description}
\item[Bart van Bouchaute] is als politicoloog verbonden aan het departement Sociologie
van de Universiteit Antwerpen, onderzoekscentrum Ongelijkheid, Armoede, Sociale
uitsluiting en de Stad (OASeS). Hij heeft o.a. onderzoek gedaan naar
stadsvernieuwing en gentrificatie in Gent.
\item[Joke van der Zwaard] is ontwikkelingspsycholoog en werkt als zelfstandig
onderzoeker/publicist op het terrein van sociale ongelijkheid, sociale
mobiliteit, emancipatie en samenlevingskwesties. Ook is ze initiatiefnemer van
o.a. De Leeszaal Rotterdam West.
\item[Reinout Kleinhans] is als stadsgeograaf verbonden aan de Faculteit Bouwkunde van
de TU Delft. Hij houdt zich o.a. bezig met wijkontwikkeling, stedelijke
herstructurering, sociale cohesie en sociaal kapitaal. E\'en van de
ontwikkelingen die hij uitgebreid heeft onderzocht is de vernieuwing in
Hoogvliet.
\end{description}

\vfill

De sprekers gaan in op de voor- en nadelen van het trekken van `kansrijken'
naar de stad. Wat betekent deze
verandering voor stad en stedelingen?

\DolFooterLarge

\newpage

\emph{Onder het motto `denken over links' biedt SP Rotterdam een podium aan
inspirerende sprekers over diverse thema's. Een lezingenserie waarin het denken
over hedendaagse sociale vraagstukken wordt geprikkeld en gescherpt. \ Deze
maand:}

\vfill

\DolMega{\textcolor{DolRed}{Stad zoekt Middenklasse}}

\vfill

\textbf{``Rotterdam begint met renovatie burgers''. Deze kop op de satirische
nieuwswebsite De Speld raakte de essentie van de kritiek op de komst van de
Markthal: dit prominente project zou niet voor Rotterdammers zijn, maar voor de
welgestelde middenklasse, voor de mensen die het college graag meer zou zien in
Rotterdam. Hetzelfde geldt voor de recente ambities voor bakfietswijken, de
bouw van luxe woontorens en aanpassing van huurwoningen ``aan de wensen van
deze tijd'': alles lijkt erop gericht de middenklasse naar de stad te trekken
en hier te houden.}

Natuurlijk staat Rotterdam niet op zichzelf: na de VINEX-vlucht is wonen in de
stad nu weer populair, en steden concurreren met elkaar om meer welgestelde
bewoners te trekken. Achter dit beleid zitten grote ambities: de middenklasse
moet ``de sociale mix versterken'', daarmee ``wijken omhoog trekken'' en tegelijk
Rotterdam tot een aantrekkelijke en draagkrachtige stad maken. In hoeverre
maakt de komst van de middenklasse deze ambities waar? En wat betekent deze
verandering voor de huidige stedelingen? Op donderdag 27 november staan deze
vragen centraal in de discussieavond Denken over Links.

De avond wordt afgetrapt met drie korte lezingen door Bart van Bouchaute, Joke
van der Zwaard en Reinout Kleinhans. Zij belichten de thematiek vanuit hun
onderzoek en ervaring in Rotterdam en andere steden. Vervolgens gaan de
sprekers verder in gesprek met elkaar en met het publiek aan de hand van een
aantal stellingen. Denk mee en zorg dat je erbij bent op 27 november!

\vfill

\emph{De eerstvolgende editie van Denken over Links heeft als thema ``Leve de
verzorgingsstaat!'' en vindt plaats op 29 januari.}

%\DolFooterLarge
\DolHline

\end{document}
