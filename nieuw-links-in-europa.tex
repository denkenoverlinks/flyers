\documentclass{article}

\usepackage{graphicx}
\usepackage{parskip}
\usepackage{color}
\usepackage{PTSansNarrow} 
\usepackage[T1]{fontenc}
\usepackage{array}
\renewcommand\familydefault\sfdefault

\pagestyle{empty}

\renewcommand\emph[1]{{\slshape#1}}

\definecolor{DolRed}{rgb}{0.92,0.11,0.15} % #eb1c26
\definecolor{DolGray}{rgb}{0.82,0.82,0.82}

\newcommand\DolHeader[1]{
\includegraphics[width=.45\columnwidth]{dol_logo_dark.png}\hfill\parbox[b]{.5\columnwidth}{#1}
\DolHline \\
{\bf Gaffelstraat 61B \DolSep Toegang gratis \DolSep Inloop: 19:30 uur \DolSep Start: 20:00 uur \DolSep Eind: 22:00 uur} \\
\DolHline
\vfill
}

\newcommand\DolHeaderLarge[1]{
\includegraphics[width=.45\columnwidth]{dol_logo_dark.png}\hfill\parbox[b]{.54\columnwidth}{#1}
\DolHline \\
{\bf Gaffelstraat 61B \DolSep Toegang gratis \DolSep Inloop 19:30 \DolSep Start 20:00 \DolSep Eind 22:00 } \\
\DolHline
}

\newcommand\DolHeaderLargeEn[1]{
\includegraphics[width=.45\columnwidth]{dol_logo_dark.png}\hfill\parbox[b]{.54\columnwidth}{#1}
\DolHline \\
{\bf Gaffelstraat 61B \DolSep Free entrance \DolSep Open 19:30 \DolSep Start 20:00 \DolSep End 22:00 } \\
\DolHline
}

\newcommand\DolHeaderEn[1]{
\includegraphics[width=.45\columnwidth]{dol_logo_dark.png}\hfill\parbox[b]{.5\columnwidth}{#1}
\DolHline \\
{\bf Gaffelstraat 61B \DolSep Free entrance \DolSep Doors open: \ 19:30 \DolSep Start: \ 20:00 \DolSep End: \ 22:00} \\
\DolHline
\vfill
}

\newcommand\DolFooter{
\vfill
\colorbox{DolRed}{\parbox{\columnwidth}{\ \parbox[b]{.88\columnwidth}{\color{white}\raggedright
{\bf `Denken over links', een maandelijkse lezingenserie in het SP-pand aan
de Gaffelstraat. Doe mee, denk mee! Iedereen is welkom.} \\[-7pt]
\hfill denkenoverlinks.nl\\[.2em]}
\hfill \includegraphics[scale=.4]{qr.png}}}
}

\newcommand\DolFooterLarge{
\vfill
\colorbox{DolRed}{\parbox{\columnwidth}{\ \ \ \parbox[b]{.84\columnwidth}{\color{white}\raggedright
{\bf `Denken over links', een maandelijkse lezingenserie in het SP-pand aan
de Gaffelstraat. Kom langs en denk mee!} \hfill denkenoverlinks.nl \\[4pt]}
\hfill \includegraphics[scale=.45]{qr.png}}}
}

\newcommand\DolFooterLargeEn{
\vfill
\colorbox{DolRed}{\parbox{\columnwidth}{\ \ \ \parbox[b]{.84\columnwidth}{\color{white}\raggedright
{\bf `Denken over links' is a monthly lecture series in the SP rooms in Gaffelstraat.
Come along and be left thinking!} \hfill denkenoverlinks.nl \\[4pt]}
\hfill \includegraphics[scale=.45]{qr.png}}}
}

\newenvironment{DolQuote}
{%
  \begin{quote}
  \begin{picture}(0,0)\put(-20,-34){\fontsize{2cm}{1em}\selectfont\bf\color{DolGray}``}\end{picture}
  \raggedright
  \setlength\parindent{1.5ex}
}{%
  \end{quote}
}

\newcommand\largeblack[2]{{\fontsize{#1}{1em}\selectfont#2}\par}

\newcommand\DolHline{\textcolor{DolRed}{\rule{\columnwidth}{6pt}}}
\newcommand\DolBullet{\textcolor{DolGray}{$\bullet$}}
\newcommand\DolSep{\hfill\DolBullet\hfill}
\newcommand\DolMega[1]{{\fontsize{1.15cm}{1em}\selectfont\bf#1}\par}
\newcommand\DolHuge[1]{\textcolor{DolRed}{\Huge\bf#1}\par}
\newcommand\DolLarge[1]{\textcolor{DolRed}{\Large\bf#1}\par}
\newcommand\DolBox[1]{\colorbox{DolRed}{\parbox{\columnwidth}{\centering\parbox{.95\columnwidth}{\rule{0pt}{13pt}\color{white}\large\bf#1\rule[-7pt]{0pt}{0pt}}}}}


\begin{document}

\DolHeader{
{\color{DolRed}\fontsize{1.25cm}{1em}\selectfont\bf
Nieuw links \\[5pt] in Europa \\
} \\
\Large {\bf Donderdag 25 september} \\
Rotterdam \\[-8pt]
}

\textbf{Donderdag 25 september is de aftrap van een nieuwe serie lezingen, waarin het 
denken over hedendaagse sociale vraagstukken wordt geprikkeld en gescherpt. Onder 
het motto Denken Over Links biedt de SP Rotterdam een podium voor inspirerende 
sprekers over diverse thema's als samenleven in de stad, racisme en de toekomst van 
de verzorgingsstaat. We trappen af met lezingen en discussie over de perspectieven van 
Europees links.}

\vfill

Bij de Europese verkiezingen in mei bleef de doorbraak van een Europees `links
van links' uit. Ondanks de crisis en de desastreuze bezuinigingspolitiek bleek
links nog onvoldoende in staat een geloofwaardig sociaal alternatief neer te
zetten. Maar er waren ook lichtpunten. Het Griekse \emph{Syriza}, het
boegbeeld van Europees radicaal links, werd de grootste partij in Griekenland.
En het nog maar vier maanden oude \emph{Podemos}, de stem van de Spaanse
`Indignados', scoorde in een klap acht procent van de stemmen in Spanje. 

Deze partijen zijn actief in landen die hard getroffen zijn door de crisis en die met grote 
werkloosheid, armoede en harde bezuinigingen te maken hebben. Zij weten met een 
radicaal programma grote groepen mensen aan zich te binden. Dat maakt nieuwsgierig 
naar het verhaal achter deze bewegingen.

Wat zijn de perspectieven van deze (relatief) nieuwe politieke bewegingen? Hoe verklaren 
ze hun succes? Hoe zit het met de verhouding tussen het activisme op straat en de 
parlementaire politiek? Deze en andere vragen komen aan de orde in twee lezingen van 
sprekers van Podemos en Syriza en de discussie die daarop volgt.

\textbf{Let op: het programma is ---bij wijze van uitzondering--- Engelstalig!}

\emph{De eerstvolgende edities van Denken over Links staan in het teken van hedendaags 
racisme (30 oktober) en gentrificatie (27 november).}

\DolFooter

\newpage

\DolHeaderEn{
{\color{DolRed}\fontsize{1.25cm}{1em}\selectfont\bf
Europe's \\[2pt] New Left \\
} \\[-3pt]
\Large {\bf Thursday September 25} \\
Rotterdam \\[-8pt]
}

\textbf{This year's edition of `Left Thinking' ---a series designed to
stimulate and focus progressive discussion on current social affairs--- kicks
off on Thursday, September 25. Presented by the Rotterdam branch of the
Socialist Party, `Denken over Links' offers a podium for inspiring and
thought-provoking speakers on such diverse issues as living together in the
city, racism and discrimination, and the future of the welfare state, amongst
many others. The 2014/15 season opens with an evening of lectures and
discussion concerning the perspectives for the Left in Europe.}

The European elections in May didn't produce a genuine, Europe-wide
breakthrough for the authentic Left. Despite the economic crisis, and
disastrous austerity measures, the Left, broadly speaking, appears as yet
unable to present a credible social alternative. But it's not all bad news:
Greece's \emph{Syriza}, the figurehead for Europe's radical left, became the
leading electoral force in that country; and \emph{Podemos}, the voice of
Spain's indignados, immediately won 8\% of the vote, despite having been formed
only four months beforehand.

These parties ---active in countries hit hard by the crisis, with soaring unemployment, 
deep poverty and far-reaching austerity--- have managed to attract wide support 
with their radical programmes. And there are some intriguing stories behind these 
results\dots

What are the perspectives for these relatively new political movements? How can 
we explain their success? How do they see the relationship between on-the-streets 
activism and parliamentary politics? These and other questions form the basis for 
lectures from representatives of both Syriza and Podemos, to be followed by an 
open discussion.

\textbf{Due to the nature of the evening, the lectures and discussion will be in English.}

\emph{Coming up in the series of Left Thinking: Modern-day Racism and Discrimination
(October 30) and Urban Gentrification (November 27).}

\DolFooter

\end{document}
